\subsection{U.S. Employment and Establishment Data}
The \href{https://www.census.gov/programs-surveys/cbp/data/datasets.html}{County Business Patterns} (CBP), \href{https://www.bls.gov/cew/downloadable-data-files.htm}{Quarterly Census of Employment and Wages} (QCEW), and \href{https://www.census.gov/data/developers/data-sets/qwi.html}{Quarterly Workforce Indicators} (QWI) are public datasets of United States employment and/or establishment counts by county FIPS codes.

The CBP provides annual employment, payroll, and establishment counts by county, with the latter further subdivided into number of establishments by establishment size (number of employees) classes.
Furthermore, this dataset groups these variables by 6-digit NAICS code.

The QCEW offers quarterly employment, wages, and establishment counts by county as well.
Unlike the CBP, however, these data do not differentiate establishment counts by establishment size classes, instead offering the total number of establishments in a county regardless of employee size.
Furthermore, this dataset also groups these variables by 4-digit NAICS code.
More granular (i.e. 6-digit) NAICS codes are available at larger geographic levels, such as by state/CSA.

Unlike the other two datasets, the QWI offers employment, hires, separations, and wages by both worker demographics (such as sex, age, and race/ethnicity) and firm characteristics (such age, size, and industry). 
However, these data do not have establishment counts and only aggregate up to 4-digit NAICS codes.

The respective granularity of the above U.S. datasets are summarized in the following table:
\begin{table}[H]
    \centering
    \caption{U.S. Data Summary}
    
    \begin{tabular}{l|c|c|c|c|c|c|c}
    \toprule
     Dataset & Frequency & NAICS & Emp. & Estabs. & Estabs. Size & Firm Size & Demographics \\
     \midrule
     CBP    & Annual & 6-digit & Yes & Yes & Yes & No & No \\
     QCEW   & Quarterly & 4-digit* & Yes & Yes & No & No & No \\ 
     QWI    & Quarterly & 4-digit & Yes & No & No & Yes & Yes \\
    \bottomrule
    \end{tabular}
    \begin{flushleft}
        \begin{footnotesize}
        * 6-digit NAICS possible if using geography larger than county
        \end{footnotesize}
    \end{flushleft}
\end{table}

\subsection{Mexico Employment and Establishment Data}
The following candidate analogous Mexico datasets all come from \href{https://en.www.inegi.org.mx/}{Instituto Nacional de Estadística y Geografía} (INEGI). 
Arguably, the most similar geographic division in Mexico to the United States county is the \href{https://en.wikipedia.org/wiki/Municipalities_of_Mexico}{municipio}, of which there are 2,478 in the country.
This narrows my search down to publicly available Mexican data that offer municipio-level employment and/or establishment counts.

The \href{https://en.www.inegi.org.mx/programas/ce/2024/#documentation}{Censos Económicos} (CE) have reasonably detailed snapshots of employment, wages, and revenues at the municipio level that are published every 5 years, with the last publication in 2024.
Their \href{https://en.www.inegi.org.mx/app/descarga/?p=3366}{pre-made tables and indicators} (i.e. summary statistics) are too coarse in either the geographic or sectoral dimension.
Fortunately, the CE have a \href{https://en.www.inegi.org.mx/app/saic/default.html}{online tool} to select various statistics (such as employment and wages) up to the 6-digit NAICS codes, which can then be exported as a table.
Once the correct parameters are chosen, the specified data can be exported as a CSV file.
The \href{https://en.www.inegi.org.mx/contenidos/app/saic/saic_historico_metodologico_ce2024.pdf}{documentation} mentions, however, that there is censoring when there are less than three establishments in the selected cell (geography $\times$ NAICS).
The CE's closest U.S. analog is the CBP as both offer employment and wage statistics up to 6-digit NAICS, although the former is five times less frequent.

The \href{https://en.www.inegi.org.mx/programas/ilmm/#documentation}{Indicadores Laborales para los Municipios de México} (ILMM) provides annual employment counts (in the informal and formal sectors) for each municipio on an annual basis.
The data also provide municipio-level demographics such as total population, population over 60, and male/female ratio.
These data, however, lack any industry deaggration, nor do they provide firm or establishment counts.
Since the ILMM provides the most aggregated statistics (such as total employment) without any industry deaggregation, it includes all municipios in Mexico without any censoring issues.
The ILMM is most similar to the U.S. QWI.

The \href{https://en.www.inegi.org.mx/app/descarga/}{Directorio Estadístico Nacional de Unidades Económicas} (DENUE) are annual records of more than 6 million business establishments in Mexico. 
Their records include the name of the establishment, NAICS code, municipio, and number of employees.
They even include latitude/longitudes of the establishments (no need for Google Maps API calls!!).
The DENUE uses the EC (mentioned above) as the main source of establishment data, which is collected via interviews and records of all active establishments during the EC census year.
In the interceding years, the DENUE uses administrative data to make updates.
The documentation states that the DENUE is not meant to be a representative survey using a subsample of Mexican establishments, rather, it is meant to be an exhaustive list of all official businesses in Mexico.
The DENUE does censor some identifiable information about the establishments (like any identifiable personal information).
This is mostly relevant for single proprietors as DENUE will censor their names.
However, geographic and NAICS codes are not censored.
Since these data do not directly provide employment counts and focus more on the establishments, these data are most analogous to the U.S. CBP.

In my opinion, I think the best use of these data would be to use both the CE and DENUE for outcome variables, and the ILMM to gather simple demographic controls.
The CE would provide municipio-level employment at the industry granularity of 6-digit NAICS codes, with the caveat that these are collected every five years.
The DENUE offers establishment counts at the municipio-level, also at 6-digit NAICS code granularity, with the added benefit of being annual data.
The ILMM would give coarse age-block demographics (such as prop. over 60) as well as population and male-female ratio.

The datasets are summarized in the following table:

\begin{table}[H]
    \centering
    \caption{Mexico Data Summary}
    
    \begin{tabular}{l|c|c|c|c|c|c|c}
    \toprule
     Dataset & Frequency & NAICS & Emp. & Estabs. & Estabs. Size & Firm Size & Demographics \\
     \midrule
     CE    & Semi-Decadal & 6-digit & Yes & No & No & No & No \\
     ILMM   & Annual & N/A & Yes & No & No & No & Yes \\ 
     DENUE    & Annual & 6-digit & No & Yes & Yes & No & No \\
    \bottomrule
    \end{tabular}
\end{table}

